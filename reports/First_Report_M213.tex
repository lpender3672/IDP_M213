\documentclass{article}
\usepackage[utf8]{inputenc}

\title{First IDP Report}
\author{Group M213 (lwp26, mah237, yhh35, yl827, yw543, yz754)}
\date{November 2022}

\begin{document}

\maketitle

\section{Introduction}
    
\section{Mechanical}
\quad The mechanical team consists of Pender, L.W. (lwp26) and Hendricks, M.A.(mah237). Our short term plan for the mechanical team was to get a good CAD model and a cardboard model out for the robot by the 14th of November. We started off by making some design decisions.

\subsection{Drive System}
\quad \quad We initially considered a few drive systems for the robot, namely a 3-wheeled differential drive system, a 4-wheeled tank drive system, making our ownn Mecanum wheels and a simple 2-wheeled system. 

\quad We ruled out the Mecanum wheels idea due to the sheer difficulty of manufacturing such a wheel without much added benefit to the project since the main priority for our team would be rapid production so that testing can begin sooner rather than later. The 3-wheeled differential drive was also ruled out due to not having 3 same-sized wheels. This is because we don't want the base of the chassis to be inclined in case we would have markers (such as QR codes) used by the software team for navigation and other purposes. The inclination would probably lead to larger errors in navigation. We ruled out the 4-wheeled tank drive system, as we were worried about slipping occuring in one of the sets of wheels if the ratios of the speeds were not accurate. This is especially important to us since we're thinking about using a light sensor as a rotary encoder to have an accurate measure of the distance travelled by the robot.

\quad Therefore, the best option we settled on was the simple 2-wheeled system, since it would be the simplest to manufacture. We plan to use the larger wheels so that our rotary encoder would be more accurate. The wheels will be connected to the higher torque lower RPM motor via the given motor adaptors. The higher torque will help in the robot going up the ramp. The motors will be attached to the robot by a metal bracket and bolting it onto the threads on the aluminium plate on the motor.

\subsection{Chassis}
\quad After looking at the sizes of the various components (such as the Arduino and the battery pack) that we needed to fit onto the chassis we make a rough guess of a dimension of the base of the chassis to be $250mm \times 140mm$. The Arduino and the battery pack will be fitted under the chassis so that the center of gravity of the robot will be lower, further reducing the chances that the robot flips when going up the ramp.

\section{Electrical}



\section{Software}


\section{Project Management}
\quad L.W. Pender was selected as the leader of the group. We created a Trello project which has functionalities like a Gantt chart and Kanban boards as our project management systems. We made a rough timeline of the large milestones of the project such as having a fully assembled robot, having the sensors all attached onto the robot and funcitoning, having a working robot and testing, and placed these onto the Trello project. We then made smaller short-term goals for each team to achieve.

\end{document}
